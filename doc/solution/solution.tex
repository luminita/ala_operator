\documentclass[12pt]{article}
\usepackage{graphicx}
\usepackage[margin=1in]{geometry}
%\usepackage{float}
\usepackage{multirow}
\usepackage{fancyhdr}
\usepackage{listings}

\title{{\bf AlaOperator - assignment solution}}

\author{
Luminita Moruz  \\
}

\begin{document}
\maketitle

\section{Problem formulation}
\label{sec:problem}
An algorithm should be implemented that takes as input a list of
operators, with each operator including a collection of offers
expressed as prefix, price. For a given phone number, the algorithm
should output the cheapest operator that has a prefix matching that
phone number. When several prefixes of an operator match a number,
the longest should be chosen.
 
\section{Implementations}
Two straightforward solutions were implemented, both briefly described
below.

\subsection{Solution 1 - simple approach}
Since each operator may potentially offer a large number of prefixes,
I chose to store their prefixes and prices as dictionaries, with
prefixes as keys and prices as values. This implies that I store the
data in a list $D = [D_1, D_2, ..., D_n]$, where $n$ is the number of
operators and each $D_i$ is a dictionary storing the prefixes and
prices of operator $i$.

\vspace{0.2cm}

Let $x=x_1 x_2 ... x_m$ be the phone number we search, where $m$ is
the number of digits. My algorithm starts by searching the full number
$x_1 .. x_m$ in all the $n$ dictionaries. In the next step, the last
digit of the number is dropped, meaning that only $x_1 .. x_{m-1}$ is
searched in the dictionaries of the operators. Following, in the next
step only $x_1 .. x_{m-2}$ is searched and so on. If at a certain
point the number is found in the dictionary $D_i$ of operator $i$,
then no search will be performed in $D_i$ in the next iterations. This
way I make sure that the longest match is returned for a given
operator. The procedure is repeated until we reached $x_1$, or until
the number was found in all $D_i, i=1..n$.

\vspace{0.2cm}

The search in a dictionary in python has in average $O(1)$
complexity. Since we have $n$ dictionaries, and a search is performed
for each prefix of the phone number, the complexity of the algorithm
would be $O(mn)$, where $m$ is the length of the phone number.

\subsection{Solution 2 - optimized approach}
When implementing the solution above, I realized that in practice the
phone number will be much longer than the prefixes. Thus, instead of
starting by searching the full word $x_1x_2 ... x_m$ in all the
dictionaries, I start instead by searching $x_1x_2 ...x_p$, where $p$
is the length of the longest prefix observed at any of the
operators. $p$ is calculated when the data is loaded.

\vspace{0.2cm}

In this case, the complexity is reduced to $O(np)$. 


\section{Limitations and possible improvements of current implementation}
The solutions provided here can be improved in various ways. Depending
on the characteristics of the input data, more efficient solutions can
be designed. Some of my own thoughts regarding these are given below.

\vspace{0.2cm}

If a very large number of operators are available, it may be more
convenient to store all the prefixes and prices in one large
dictionary. This way, instead of searching in several dictionaries, we
search in just one dictionary. This would also be more space
efficient, since a prefix is stored just once.

\vspace{0.2cm} 

Also, if the number of phone numbers to be searched is very large,
then a dictionary including solutions for the prefixes searched up to
that point can be maintained. Note that this would imply that a search
in yet another dictionary is to be performed.

\vspace{0.2cm}

The current implementation is a small proof-of-principle. As an
example, no proper validation of the input data is performed, which
would be a must in a real world application. In addition, my program
does not implement a proper exception handling, neither it provides
informative error messages. It rather relies on the user to provide
correct input data :).

\vspace{0.2cm} 

Regarding the unit tests, they are by no means complete. In addition,
they make use of real files when testing some of the functions. In a
real application, this may slow down the tests significantly. Instead,
mock objects could be used.

\vspace{0.2cm} 

The system tests evaluate the functionality of the program for only a
few simple examples. In reality, one should test the program for more
complex input. Also, it should be tested that the correct error
messages are returned when erroneous input data is provided.

\end{document}

